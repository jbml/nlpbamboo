\documentclass[12pt,a4paper]{article}
\def\pgfsysdriver{pgfsys-dvipdfm.def}

% 加载宏包,格式设定
\usepackage{fontspec,indentfirst}
\usepackage{xunicode}               % unicode macros
\usepackage{xltxtra}                % fixes/extras
\usepackage[margin=1.5cm] {geometry}
\usepackage{tikz}
\usepackage[ruled,vlined]{algorithm2e}

\XeTeXlinebreaklocale "zh"
\XeTeXlinebreakskip = 0pt plus 1pt minus 0.1pt

% 字体
\newfontfamily\mono{Courier New}
\newfontfamily\serif{Times New Roman}
\newfontfamily\hei{"黑体"}
\setmainfont{"宋体"}
% for algorithm2e
%\SetAlFnt{\serif\em}
%\SetAlCapFnt{\serif}
%\SetAlTitleFnt{\serif}

% 行间距
\renewcommand{\baselinestretch}{1.2}

% 名词定义
\def\maxforward{最大正向匹配{\serif(Maximum Forward Matching)}}
\def\maxbackward{最大逆向匹配{\serif(Maximum Backward Matching)}}
\def\NGram{N-元语法模型{\serif(N-Gram Model)}}
\def\UniGram{一元语法模型{\serif(UniGram Model)}}

% 标题定义
\title{常用中文分词算法简介}
\author{杨家宁}

% 文章开始
\begin{document}
\maketitle

\def\abstractname{\hei 摘~~要}
\begin{abstract}
中文分词的研究时至今日已经产生出了很多种方法。
例如:\maxforward 、\maxbackward 还有基于统计语言模型的\NGram 方法。
其中\NGram 具有较高的分词准确度,但实现起来确相对复杂。
而\maxforward 方法虽然准确度不高,但由于实现简单因此有比较广泛的应用。
本文将先简单介绍\maxforward 和\maxbackward 算法以及他们的一些问题。
再着重介绍\NGram 中比较简单的一种模型\UniGram 。
最后,再介绍一些对模型的修正算法。
\end{abstract}

\section{\maxforward}
\begin{table}
\caption{例如:切分句子 “我喜欢奥运会”}
\label{tbl_maxforward}
	\begin{center}
		\begin{tabular}{l|c|l|l}
			当前处理字符串 & 是否匹配    & 目前最长词 & 当前分词结果   \\
			\hline
			我\$           & {\serif Y}  & 我         &                \\
			我喜\$         & {\serif N}  & 我         &                \\
			我喜欢\$       & {\serif N}  & 我         &                \\
			我喜欢奥\$     & {\serif N}  & 我         &                \\
			我喜欢奥运\$   & {\serif N}  & 我         &                \\
			我喜欢奥运会\$ & {\serif N}  & 我         &                \\
			喜\$           & {\serif Y}  & 喜         & 我             \\ 
			喜欢\$         & {\serif Y}  & 喜欢       & 我             \\ 
			喜欢奥\$       & {\serif N}  & 喜欢       & 我             \\ 
			喜欢奥运\$     & {\serif N}  & 喜欢       & 我             \\ 
			喜欢奥运会\$   & {\serif N}  & 喜欢       & 我             \\ 
			奥\$           & {\serif Y}  & 奥         & 我 喜欢        \\ 
			奥运\$         & {\serif Y}  & 奥运       & 我 喜欢        \\ 
			奥运会\$       & {\serif Y}  & 奥运会     & 我 喜欢        \\ 
		    \$             &             &            & 我 喜欢 奥运会 \\
		\end{tabular}
	\end{center}
\end{table}

我们先来讲解\maxforward 算法。
最大正向顾名思义就是从句子开头向结尾搜索,使得每次切分出来的词的长度最大。
表\ref{tbl_maxforward}一步步演示了\maxforward 的切分过程。算法\ref{alg_maxforward}具体描述了这个算法。
\incmargin{1em}
\begin{algorithm}
	\label{alg_maxforward}
	\dontprintsemicolon
	\linesnumbered
	\SetKwData{Token}{token}
	\SetKwData{MaxTokenSize}{max}
	\SetKwData{Result}{result}
	\SetKwFunction{Strlen}{strlen}
	\SetKwFunction{Substr}{substr}
	\SetKwFunction{Concat}{concat}
	\SetKwInOut{Input}{Input}
	\SetKwInOut{Output}{Output}

	\Input{A string in chinese $s$}
	\Output{A string after Segmentation $t$}
	\Begin{
		\For{$i\leftarrow 0$ \KwTo \Strlen{$s$}}{
			\For{$j\leftarrow$ \MaxTokenSize \KwTo $0$} {
				\Token$\leftarrow$\Substr{$s$, $i$, $j$}\;
			}
			\Concat{$t$, \Token}\;
			\Concat{$t$, $" "$}\;
			$i\leftarrow i + j - 1$\;
		}
	}
	\caption{\maxforward}
\end{algorithm}
\decmargin{1em}

\end{document}
