\documentclass[12pt,a4paper]{article}
\def\pgfsysdriver{pgfsys-dvipdfm.def}

% 加载宏包
\usepackage{fontspec,indentfirst}
\usepackage{xunicode}               % unicode macros
\usepackage{xltxtra}                % fixes/extras
\usepackage[margin=1.5cm] {geometry}
\usepackage{tikz}
\usepackage[ruled,vlined]{algorithm2e}
\usepackage[below]{placeins} % 处理浮动图像
\usepackage{zhspacing}
\zhspacing

% 行间距
\renewcommand{\baselinestretch}{1.2}

% 名词定义
\def\maxforward{最大正向匹配(Maximum Forward Matching)}
\def\maxbackward{最大逆向匹配(Maximum Backward Matching)}
\def\NGram{N-元语法模型(N-Gram Model)}
\def\UniGram{一元语法模型(Unigram Model)}
\def\DijkstraAlgorithm{迪克斯特拉算法(Dijkstra's Algorithm)}

% 标题定义
\title{常用中文分词算法简介}
\author{杨家宁}

% 文章开始
\begin{document}

\maketitle

\def\abstractname{\bf 摘 要}
\begin{abstract}
中文分词的研究时至今日已经产生出了很多种方法。例如:\maxforward 、\maxbackward
还有基于统计语言模型的\NGram 方法。其中\NGram 具有较高的分词准确度,但实现起来
确相对复杂。而\maxforward 方法虽然准确度不高,但由于实现简单因此有比较广泛的应
用。本文将先简单介绍\maxforward 和\maxbackward 算法以及他们的一些问题。再着重介
绍\NGram 中比较简单的一种模型\UniGram 。最后,再介绍一些对模型的修正算法。
\end{abstract}

\section{\maxforward}
\begin{table}
\caption{\maxforward 算法切分句子 “我喜欢奥运会”}
\label{tbl_maxforward}
	\begin{center}
		\begin{tabular}{l|c|l|l}
			当前处理字符串 & 是否匹配    & 目前最长词 & 当前分词结果   \\
			\hline
			我\$           &  Y  & 我         &                \\
			我喜\$         &  N  & 我         &                \\
			我喜欢\$       &  N  & 我         &                \\
			我喜欢奥\$     &  N  & 我         &                \\
			我喜欢奥运\$   &  N  & 我         &                \\
			我喜欢奥运会\$ &  N  & 我         &                \\
			喜\$           &  Y  & 喜         & 我             \\ 
			喜欢\$         &  Y  & 喜欢       & 我             \\ 
			喜欢奥\$       &  N  & 喜欢       & 我             \\ 
			喜欢奥运\$     &  N  & 喜欢       & 我             \\ 
			喜欢奥运会\$   &  N  & 喜欢       & 我             \\ 
			奥\$           &  Y  & 奥         & 我 喜欢        \\ 
			奥运\$         &  Y  & 奥运       & 我 喜欢        \\ 
			奥运会\$       &  Y  & 奥运会     & 我 喜欢        \\ 
		    \$             &          &            & 我 喜欢 奥运会 \\
		\end{tabular}
	\end{center}
\end{table}

我们先来讲解\maxforward 算法。所谓“最大正向”,顾名思义就是从句子开头向结尾搜
索,使得每次切分出来的词的长度最大。表\ref{tbl_maxforward}一步步演示了
\maxforward 的切分过程。详细的算法描述,见算法\ref{alg_maxforward}。

\begin{algorithm}
	\label{alg_maxforward}
	\dontprintsemicolon
	\linesnumbered
	\SetKwData{Token}{token}
	\SetKwData{MaxTokenSize}{max}
	\SetKwData{Result}{result}
	\SetKwFunction{Strlen}{strlen}
	\SetKwFunction{Substr}{substr}
	\SetKwFunction{Concat}{concat}
	\SetKwFunction{Find}{find}
	\SetKwInOut{Input}{Input}
	\SetKwInOut{Output}{Output}

	\Input{A string in chinese $s$}
	\Output{A string after Segmentation $t$}
	\Begin{
		\For{$i\leftarrow 0$ \KwTo \Strlen{$s$}}{
			\For{$j\leftarrow$ \MaxTokenSize \KwTo $0$} {
				\Token$\leftarrow$\Substr{$s$, $i$, $j$}\;
				\If{\Find{\Token}} {
					break\;
				}
			}
			\Concat{$t$, \Token}\;
			\Concat{$t$, $" "$}\;
			$i\leftarrow i + j - 1$\;
		}
	}
	\caption{\maxforward}
\end{algorithm}

\maxforward 虽然简单,但很多时候会因为切词过长导致错误。
例如,句子“研究生命起源”,会被\maxforward 切分成 “研究生 命 起源”。显然是错误的。

\section{\maxbackward}

现在让我们把\maxforward 方法反过来,也就是从后向前反向扫描句子。\ref{tbl_maxbackward}向我们展示了这个过程。

\begin{table}
\caption{\maxbackward 算法切分句子 “研究生命起源”}
\label{tbl_maxbackward}
	\begin{center}
		\begin{tabular}{l|c|l|l}
			当前处理字符串           & 是否匹配 & 目前最长词 & 当前分词结果   \\
			\hline
			源\$                     &  Y  & 源         &                \\
			起源\$                   &  Y  & 起源       &                \\
			命起源\$                 &  N  & 起源       &                \\
			生命起源\$               &  N  & 起源       &                \\
			研究生命起源\$           &  N  & 起源       &                \\

			命\$                     &  Y  & 命         & 起源           \\
			生命\$                   &  Y  & 生命       & 起源           \\
			究生命\$                 &  N  & 生命       & 起源           \\
			研究生命\$               &  N  & 生命       & 起源           \\

			究\$                     &  Y  & 究         & 生命 起源      \\
			研究\$                   &  Y  & 研究       & 生命 起源      \\

			\$                       &          &            & 研究 生命 起源 \\
		\end{tabular}
	\end{center}
\end{table}

这种把\maxforward 扫描方向倒转过来的方法被称为\maxbackward 虽然\maxbackward 把
上面这句\maxforward 切分错误的句子正确切分,但是我们不难发现单纯翻转扫描方式还
是不能解决切词过长的问题,例如:“欢迎来北京”,会被\maxbackward 切分成“欢 迎来 北京”。

虽然\maxforward 和\maxbackward 存在都切分过长导致错误的问题,但在实践中
\maxbackward 的准确率略高于\maxforward

\section{\UniGram}

下面我们来介绍一种稍微复杂一点的算法\UniGram,他有比\maxforward 和\maxbackward
高得多的准确率。

\UniGram 是\NGram 的一种,也是其中最简单的一种。\UniGram 算法力图找到一种切分,
让被切分后的句子中所有词的概率总和最大。下面我们介绍一种称作\DijkstraAlgorithm
的算法来高效地解决这个问题。\footnote{直接而草率的思维会指导我们枚举所有的切分
可能,然后逐个计算每种切分里所有词的概率总和。但是只要稍加计算我们就会发现,一
个含有32个字的句子大约会产生约20亿种切分。即使对于计算机说来遍历这些切分也需要
很大的代价。}

事实上,我们可以把分词看成是一个最短路问题。把切分点\footnote{包括句子最左边和最右边的空位}看作
是节点,把两个切分点之间词的频率看作距离。那么寻找最大概率切分这个问题就被转化
成了从最左点到最右点的最短路问题\footnote{就数值上而言,这个问题应该被称作个最长路问题。
但我们还是习惯性地称其为最短路问题} 

\DijkstraAlgorithm是一种动态规划算法

\begin{algorithm}
	\label{alg_maxforward}
	\dontprintsemicolon
	\linesnumbered
	\SetKwData{Token}{token}
	\SetKwData{MaxTokenSize}{max}
	\SetKwData{Result}{result}
	\SetKwData{BackRef}{backref}
	\SetKwData{Score}{score}
	\SetKwData{Prob}{prob}
	\SetKwFunction{Strlen}{strlen}
	\SetKwFunction{Substr}{substr}
	\SetKwFunction{Concat}{concat}
	\SetKwFunction{GetProb}{probability}
	\SetKwInOut{Input}{Input}
	\SetKwInOut{Output}{Output}

	\Input{A string in chinese $s$}
	\Output{A string after Segmentation $t$}
	\Begin{
		\Score $\leftarrow [0, 0, 0, ...]$ \;
		\BackRef $\leftarrow [0, 0, 0, ...]$ \;
		\For{$i\leftarrow 0$ \KwTo \Strlen{$s$}}{
			\For{$j\leftarrow$ $0$ \KwTo \MaxTokenSize} {
				\Token $\leftarrow$ \Substr{$s$, $i$, $j$}\;
				\Prob $\leftarrow$ \GetProb{\Token}\;
				\If {\Score$[ i + j ]$  $<$ \Score$[ i ]$ $ + $ \Prob } {
					\Score$[ i + j ]$ $\leftarrow$ \Score$[i]$ + \Prob \;
					\BackRef$[i]$ $\leftarrow i$;
				}
			}
		}
		\While{$i > 0$} {
			\Token$\leftarrow$\Substr{$s$, \BackRef$[i]$, $i - $ \BackRef$[i]$}\;
			\Concat{$t$, \Token}\;
			\Concat{$t$, $" "$}\;
			$i \leftarrow$ \BackRef$[i]$
		}
	}
	\caption{\maxforward}
\end{algorithm}




\end{document}
